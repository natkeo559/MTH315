\documentclass[a4paper]{article}
\usepackage[shortlabels]{enumitem}
\usepackage{amsmath,amssymb,amsthm,amsfonts,tikz}
\newcommand{\bb}{\textbf}
\newcommand{\real}{\mathbb{R}}
\newcommand{\nat}{\mathbb{N}}
\newcommand{\cir}[1]{\tikz[baseline]{%
    \node[anchor=base, draw, circle, inner sep=0, minimum width=1.2em]{#1};}}

\begin{document}

    \begin{center}
        \Large \textbf{10/19} \\
        \large \textit{Nathan Keough} \\
        Acknowledgements: \emph{Utilized the study session: Worked independently and checked work cooperatively with other classmates on this practice set.} \vspace{.5pc} \\ \line(1,0){300} 
        \vspace{1pc}
    \end{center} 
    
    \begin{flushleft}

        % % % % % % % % % % % % % % % % % %

        \textbf{Wade Exercise 2.3.6} \\
        This result is used in Section 6.3 and elsewhere.  \\
        \begin{enumerate}[a.]
            \item Suppose that \{$x_n$\} is a monotone increasing sequence in $\real$ (not necessarily bounded above). Prove that there is an extended real number $x$ such that $x_n \to x$ as $n \to \infty$. 
            \item State and prove an analogous result for decreasing sequences. 
        \end{enumerate} 


        \begin{proof}\mbox{}\\*

            \begin{tabbing}
                a. \= Let {$x_n$} be be a monotone increasing sequence. \\
                   \> If {$x_n$} is bounded above, then $\exists x \in \real \ni x_n \to x$ [MCT]. \\
                   \> If {$x_n$} is not bounded, then given any upper bound $M > 0$, $\exists N \in \nat \ni x_N > M$. \\
                   \> So, {$x_n$} is increasing and $n \geq N \implies x_n \geq x_N > M$. \\
                   \> Thus if {$x_n$} is increasing and not bounded above, then {$x_n$} diverges to $\infty$ as $n \to \infty$. \\
                          \mbox{}\\*
                b. \= Likewise, $-${$x_n$} is a decreasing sequence. \\
                   \> If $-${$x_n$} is bounded below, then $\exists x \in \real \ni -x_n \to -x$ [MCT]. \\
                   \> If $-${$x_n$} is not bounded, then given any lower bound $M < 0$, $\exists N \in \nat \ni x_N < M$. \\
                   \> So, $-${$x_n$} is decreasing and $n \leq N \implies x_n \leq x_N < M$. \\
                   \> Thus if $-${$x_n$} is decreasing and not bounded below, then $-${$x_n$} diverges to $-\infty$ as $n \to \infty$.
            \end{tabbing}
            
        \end{proof}

    \end{flushleft}

\end{document}