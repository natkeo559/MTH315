\documentclass[a4paper]{article}
\usepackage[shortlabels]{enumitem}
\usepackage{amsmath,amssymb,amsthm,amsfonts,tikz}
\newcommand{\bb}{\textbf}
\newcommand{\real}{\mathbb{R}}
\newcommand{\cir}[1]{\tikz[baseline]{%
    \node[anchor=base, draw, circle, inner sep=0, minimum width=1.2em]{#1};}}

\begin{document}

    \begin{center}
        \Large \textbf{9/21} \\
        \large \textit{Nathan Keough} \\
        Acknowledgements: \emph{Utilized the study session, recieved help on 1.3.7 from other students, proofs finished independently.} \vspace{.5pc} \\ \line(1,0){300} 
        \vspace{1pc}
    \end{center} 
    
    \begin{flushleft}
        \textbf{Wade 1.3.4} \\
        Prove that a lower bound of a set need not be unique but the infimum of a given set $E$ is unique. 

            \begin{proof}\mbox{}\\*
                We must show that:\\

                \begin{align*}
                    \cir{1}& \text{ Lower bound need not be unique.} \\
                    \cir{2}& \text{ } \inf \text{ are unique.} \\
                \end{align*}
                
                \begin{tabbing}
                    $\cir{1}$ \= Consider a set $E \subseteq \real \ni E \neq \varnothing$ and $E$ is bounded below, \\
                              \> so, $\exists m \in \real \ni x \geq m \forall x \in E$. \\
                              \> Now, consider $n \leq m$, thus $\forall x\in E, x \geq n$. \\
                              \> It has been shown that two lower bounds of a set E may exist, \\
                              \> $\therefore$ a lower bound for a set need not be unique. \\

                    $\cir{2}$ \= Assume there exists two $\inf$ of a set $E \subseteq \real$, $i_1, i_2$ respectively \\
                              \> whereby both are lower bounds of $E$. Then, $i_1 \leq x$ and $i_2 \leq x \forall x\in E$. \\
                              \> By definition of infimum, $i_1 \leq i_2 \leq x$ and $i_2 \leq i_1 \leq x$ must be true. \\ 
                              \> The only case where this holds is when $i_1 = i_2$. \\
                              \> $\therefore$ the infimum of a given set E is unique.
                \end{tabbing}
                
            \end{proof}

        \textbf{Wade 1.3.7} \\
        \begin{enumerate}[a.]
            \item Prove that if $x$ is an upper bound of a set $E \subseteq \real$ and $x\in E$, then $x$ is  the supremum of $E$.
                \begin{proof}\mbox{}\\*
                    Consider a nonempty set $E\subseteq \real$ that is bounded above and let $x\in E$. \\
                    Let $x \in m \forall$ upper bounds $m$ of $E$. \\
                    $x \in m \implies \forall y\in E, y \leq x$. \\
                    Since $x \in E$, $x \leq m$
                    By definition of supremum, $y \leq x \leq m $ $\implies \sup E = x$. \\
                    $\therefore$ if $x$ is an upper bound of a set $E \subseteq \real$ and $x\in E$, then $x$ is  the supremum of $E$.
                \end{proof}
            \item Make and prove an analogous statement for the infimum of $E$.
                

                \begin{proof}\mbox{}\\*
                    Consider a nonempty set $E\subseteq \real$ that is bounded above and let $x\in E$. \\
                    Let $x \in n \forall$ upper bounds $n$ of $E$. \\
                    $x \in n \implies \forall y\in E, y \geq x$. \\
                    Since $x \in E$, $x \geq n$
                    By definition of infimum, $x \geq y \geq n $ $\implies \inf E = x$. \\
                \end{proof}

            \item Show by example that the converse of each of these statements is false. 
            
            Take for example, the set $E = (0, 1)$. Neither $0$ nor $1$ are in the set, \\
            yet the infinum and supremum of $E$ is $\inf E = 0, \sup E = 1$ \\
            $\therefore$ since $\inf E = 0, \sup E = 1$ and $0,1 \notin E$, the converse of these statement must be false.

        \end{enumerate}
        
    \end{flushleft}

\end{document}