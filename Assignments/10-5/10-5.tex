\documentclass[a4paper]{article}
\usepackage[shortlabels]{enumitem}
\usepackage{amsmath,amssymb,amsthm,amsfonts}
\newcommand{\bb}{\textbf}
\newcommand{\real}{\mathbb{R}}
\newcommand{\nat}{\mathbb{N}}

\begin{document}

    \begin{center}
        \Large \textbf{10/5} \\
        \large \textit{Nathan Keough} \\
        Acknowledgements: \emph{Utilized the study session: Worked independently and checked work cooperatively with other classmates on this practice set.} \vspace{.5pc} \\ \line(1,0){300} 
        \vspace{1pc}
    \end{center} 
    
    \begin{flushleft}

        % % % % % % % % % % % % % % % % % %

        \textbf{Wade 2.1.1a} \\
        $2 - \frac{1}{n}\rightarrow 2\ as \ n \to \infty$ 
        \begin{proof}\mbox{}\\*
            Let $\epsilon > 0$. By the AP, $\exists N \in \nat \ni$ \\
            $N > \frac{1}{\epsilon}$. Then $n \geq N \implies n > \frac{1}{\epsilon} \implies \frac{1}{n} < \epsilon$.\\
            So $n \geq N \implies |2 - \frac{1}{n} -2| = \frac{1}{n} < \epsilon$. \\
            $\therefore 2 - \frac{1}{n} \to 2 \ as \ n \to \infty$. 
            
        \end{proof}

        % % % % % % % % % % % % % % % % % %

        \textbf{Wade 2.1.1c} \\
        $3 \left(1 + \frac{1}{n}\right) \rightarrow3\ as\ n \to \infty$
        \begin{proof}\mbox{}\\*
            Let $\epsilon > 0$. By the AP, $\exists N \in \nat \ni$ \\
            $N > \frac{3}{\epsilon}$. Then $n \geq N \implies n \geq \frac{3}{\epsilon} \implies \frac{3}{n} < \epsilon$.\\
            So $n \geq N \implies |3\left(1+\frac{1}{n}\right) -3| = \frac{3}{n} < \epsilon$. \\
            $\therefore 3 \left(1 + \frac{1}{n}\right) \to 3$. 

            
        \end{proof}

        % % % % % % % % % % % % % % % % % %

        \textbf{Wade 2.1.2a} \\
        $1 + 2x_n \rightarrow 3\ as\ n \rightarrow \infty$     
        \begin{proof}\mbox{}\\*
            Let $\epsilon > 0$. Then since $X_n \to 1$, $\exists N \in \nat \ni$ \\
            $n \geq N \implies |X_n -1| < \frac{\epsilon}{2}$ \\
            Then $n \geq N \implies$
            \begin{align*}
                & 2|X_n -1| \\
                &= |2||X_n -1| \\
                &= |2(X_n -1)| \\
                &= |2X_n -2| \\
                &= |1 + 2X_n - 3| < \epsilon.
            \end{align*}
            $\therefore 1 + 2X_n \to 3$

        \end{proof}

        % % % % % % % % % % % % % % % % % %

        \textbf{Wade 2.1.2b} \\
        $\frac{\pi x_n - 2}{x_n} \rightarrow \pi -2\ as\ n \rightarrow \infty$            
        \begin{proof}\mbox{}\\*
            Let $\epsilon > 0$. Then since $X_n \to 1$, $\exists N_1 \in \nat \ni$ \\
            $n \geq N_1 \implies |X_n -1| < \frac{\epsilon}{2X_n}$ \\
            Let $\epsilon = 0.5$ so $\exists N_2 \in \nat \ni n \geq N_2 \implies |X_n -1| < 0.5$ \\
            So then, $n \geq N_2 \implies X_n > 0.5$. \\
            Now let $N = MAX\{N_1, N_2 \}$. \\
            Hence, $n \geq N \implies$
            \begin{align*}
                n \geq N_1 \\
                & \implies |X_n -1| < \frac{\epsilon}{2X_n} \\
                & \implies n \geq N_2 \\
                & \implies 0<\frac{1}{2X_n} < \frac{1}{2} \\
                Then \\
                & \left| \frac{\pi X_n -2}{\pi} - \pi - 2\right| \\
                & = \frac{|2X_n -1|}{X_n} < \epsilon\\
                & \implies |X_n -1| < \frac{\epsilon}{2X_n} \forall n \geq N.
            \end{align*}
            $\therefore \frac{\pi x_n - 2}{x_n} \to \pi -2$
            
        \end{proof}

        % % % % % % % % % % % % % % % % % %

    \end{flushleft}

\end{document}