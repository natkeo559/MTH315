\documentclass[a4paper]{article}
\usepackage[shortlabels]{enumitem}
\usepackage{amsmath,amssymb,amsthm,amsfonts}
\newcommand{\bb}{\textbf}
\newcommand{\real}{\mathbb{R}}

\begin{document}

    \begin{center}
        \Large \textbf{9/14} \\
        \large \textit{Nathan Keough} \\
        Acknowledgements: \emph{Utilized the study session: Worked independently and checked work cooperatively with other classmates on this practice set.} \vspace{.5pc} \\ \line(1,0){300} 
        \vspace{1pc}
    \end{center} 
    
    \begin{flushleft}
        \begin{enumerate}
            \item
            Wade Exercise 1.3.0 Part a
            This is false: \\

            \begin{proof}
                Proof by counterexample: \\ 

                If $A\cap B = \varnothing \implies \sup (A\cap B) = DNE$.
                
            \end{proof}

            \item
            Wade Exercise 1.3.0 Part b
            \begin{proof}
                Because $A\neq \varnothing$, and bounded above, $\exists \sup (A)$ by the Completeness Axiom. \\
                We must show: \\
                a. $\forall y\in B, y = \varepsilon \sup (A)$. \\
                b. $forall$ upper bounds $N$ of $B$, $\varepsilon \sup (A) \leq N$. 

                \begin{enumerate}
                    \item
                    Let $y \in B$. Then $y=\varepsilon a$ for \\
                    some $a \in A$. \\
                    We know $a \leq \sup (A)$. \\
                    so by MP i, $\varepsilon a \leq \varepsilon \sup (A)$ \\
                    Thus $y \leq \varepsilon \sup (A) \forall y \in B$, \\ 
                    $\therefore \varepsilon \sup (A)$ is an upper bound of $B$.

                    \item
                    Recall $\forall$ upper bound $M$ of $A$, $\sup A \leq M$. \\
                    Also notice that by definition of supremum, $a\leq \sup A \leq M$. \\
                    $a\leq \sup A \leq M \implies \varepsilon a\leq \varepsilon \sup A \leq \varepsilon M \implies \varepsilon a\leq \varepsilon M$ \\
                    $\implies y \leq \varepsilon M$. \\
                    So, $\varepsilon M$ is an upper bound of $B$, and $\varepsilon M \subseteq N$.\\
                    $\therefore \sup A \leq M$, thus $\varepsilon \sup A \leq N$. \\
                \end{enumerate}
                Since both a and b were shown, If $A\neq \varnothing$ and is a bounded subset of $\real$ and $B=\{ \varepsilon x : x \in A\}$, then $\sup B =  \varepsilon \sup A$
                
            \end{proof}
            

            \item
            Wade Exercise 1.3.1 (no supporting work required)
            \begin{enumerate}
                \item
                $ E = \{x\in \real : x^2 +2x = 3\}$ \\
                $\sup E = 1$ \\ 
                $\inf E = -3$

                \item
                $ E = \{x\in \real : x^2 -2x + 3 > x^2 \text{ and } x>0\}$\\
                $\sup E = 3/2$ \\
                $\inf E = 0$


                \item
                $ E = \{p/q \in \mathbb{Q} : p^2 < 5q^2 \text{ and } p,q>0 \}$\\
                $\sup E = \sqrt{5}$ \\
                $\inf E = 0$

                \item
                $ E = \{x \in \real : x = 1+ {-1}^n / n \text{ for }n\in \mathbb{N} \}$\\
                $\sup E = 3/2$ \\
                $\inf E = 0$

                \item
                $ E = \{x \in \real : x = 1/n + {-1}^n \text{ for }n\in \mathbb{N} \}$\\
                $\sup E = 3/2$ \\
                $\inf E = 0$

                \item
                $E = \{2 - {(-1)}^n / n^2 : n\in \mathbb{N} \}$\\
                $\sup E = 3$ \\ 
                $\inf E = 7/4$

            \end{enumerate}
            
        \end{enumerate}
    \end{flushleft}

\end{document}